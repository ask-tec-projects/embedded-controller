\chapter{Interfaces}

\section{REST API}

The Metronome application provides a RESTful HTTP API for accessing the current frequency. The application is pre-configured to serve this service on IP address 192.168.1.1.

The returned frequency represents the number of on/off cycles the led will perform every second.

It can then be accessed with an HTTP client; this example makes use of cURL:

\begin{verbatim}
    $ curl -X GET -H 'Accept: application/json' http://192.168.1.1/frequency
    {"frequency":8}
\end{verbatim}

Accessing any other resource will result in an HTTP 404 error:

\begin{verbatim}
    $ curl -X GET -H 'Accept: application/json' http://192.168.1.1/
    Not Found
\end{verbatim}


\subsection{Configuring client connectivity}

In order for a client to connect to the application an ethernet cord must be connected between the application hardware and the client hardware. An example to configure a valid IP address for the client on a modern distribution of GNU/Linux is shown below:

\begin{verbatim}
    $ sudo ifconfig <device_name> 192.168.1.2
\end{verbatim}

Available devices can be accessed with:

\begin{verbatim}
    $ ip -brief a
\end{verbatim}

\section{Hardware}

The hardware provides an interface for controlling the frequency through physical buttons.

The decrement button (shown with a purple wire) reduces the frequency of the led, while the increment button (shown with a yellow wire) increases the frequency.

The rate of change is pre-configured in the application.
